\documentclass[a4paper,14pt]{article}
\usepackage{tempora} %Times New Roman alike
\usepackage[14pt]{extsizes}
\usepackage{cmap}     
\usepackage{mathtext}
\usepackage[T2A]{fontenc}
\usepackage[utf8]{inputenc}
\usepackage[english,russian]{babel}
\usepackage{indentfirst}
\usepackage{url}
\usepackage[left=1.5cm,right=1.5cm,top=1.5cm,bottom=2cm]{geometry}
\frenchspacing

\author{Андрюхин Борис БЭАД223}
\title{Гендерное неравенство на рынке труда в России}
\date{\today}

\begin{document}

\maketitle

\section{К вопросу о разновидностях трудовой гендерной сегрегации Воронов А.А.}
В этой статье хорошо описана и проиллюстрирована сегрегация на российском рынке труда. Также определены основные виды сегрегации и причины по которым может создаваться неравенство в той или иной отрасли.
В статье выявлены некоторые ключевые факторы, которые влияют на гендерные различия в оплате труда.  Например, одним из факторов является вертикальная сегрегация, при которой женщины часто занимают менее оплачиваемые и низкоквалифицированные должности по сравнению с мужчинами. Это может быть связано с наличием стереотипов и предрассудков, а также препятствиями в повышении профессионального статуса.
Другим фактором является горизонтальная сегрегация, когда женщины склонны работать в отдельных отраслях или сферах, которые обычно характеризуются низкими заработными платами. Гендерные стереотипы и неравное распределение обязанностей в домашней сфере также могут влиять на выбор профессии и карьерные возможности женщин. 
К допущениям стоит отнести:

\begin{itemize}
    \item Монолитность групп: Допущение, что все мужчины и все женщины имеют одинаковые опыты и возможности, может быть неверным. Внутри каждой группы есть многочисленные различия и неравенства, связанные с расовой принадлежностью, этническими факторами, социальным статусом и другими переменными.;
    \item Двухполовая модель: Исследование ориентируются на традиционную двухполовую модель, разделяя людей только на мужчин и женщин. Это предположение упрощает анализ, но может игнорировать разнообразие гендерной идентичности и неучтенные группы, такие как непроявляющие явно свой пол или трансгендерные люди. (это можно считать допущением, однако процент людей, которые относятся к неучтенным группам мал и не может кардинально повлиять на результаты);
\end{itemize}

 \url{https://cyberleninka.ru/article/n/k-voprosu-o-raznovidnostyah-trudovoy-gendernoy-segregatsii}



\section{Гендерные различия в оплате труда в России Ощепков А.Ю.}
В этой статье качественно описываются причины, по которым женщинам обычно платят меньше, а именно различия в человеческом капитале6 дискриминация женщин на рынке труда и концентрация женщин и мужчин в разных отраслях экономики. В этой статье также упоминается вертикальная сегрегация, как один из факторов более низких доходов у женщин, что коррелирует с первой статьей. К особенностям статьи можно отнести учет дискриминации женщин, что часто относится к допущениям в исследованиях. Важным моментом статьи является упоминания особенностей Российского рынка труда, которые достались от СССР. 
Также хочется отметить дифференциацию в вычислениях, что помогает достаточно четко рассмотреть проблему в деталях.
К допущениям стоит отнести:
\begin{itemize}
    \item Неучет неформальной экономики и незарегистрированных рабочих мест: Официальные данные исследований зачастую не учитывают неформальную экономику и незарегистрированных работников. Это может привести к недооценке масштабов гендерных различий в оплате труда, поскольку многие женщины работают на неофициальных работах с низкой оплатой.
\end{itemize}
\url{https://cyberleninka.ru/article/n/gendernye-razlichiya-v-oplate-truda-v-rossii/viewer }
\section{Экономические причины гендерного неравенства в России Тагаров Б.Ж.}
В России никто не проводит исследования продуктивности труда мужчин и женщин в целом и по отраслям, а это один из ключевых показателей, который мог бы объяснить неравенство на рынке труда. В этой же статье упоминается работа Г. Беккера об эффективности гендерного разделения труда, которое выбирается в каждой семье. В отличие от многих он рассматривал именно семьи, а не отдельных индивидов. Зачастую для максимизации доходов семьи мужчина предпочтительнее в качестве рыночной занятости, так как имеет сравнительные преимущества над женщиной, в том числе естественные (не надо тратить время на рождение детей и уход), а также мужчина в среднем имеет большую физическую силу и более востребован на рынке труда, где это необходимо. Я считаю это стоило отметить, так как эффективность разделения не учитывалась во всех остальных статьях.
К допущениям стоит отнести:
\begin{itemize}
    \item Учет только средней заработной платы: Исследование сосредотачивается на средней заработной плате мужчин и женщин, что может упростить анализ, но не учитывать внутригрупповые различия. Имеет значение учитывать, какая доля мужчин и женщин получает высокие или низкие заработные платы.
\end{itemize}
\url{https://cyberleninka.ru/article/n/ekonomicheskie-prichiny-gendernogo-neravenstva-na-rynke-truda/viewer}
\section{Гендерные стереотипы на рынке труда Ярославцева У.А.}
В этой статье автор глубоко погружается в тему гендерных стереотипов и трудового законодательства РФ, которое должно гарантировать равенство мужчин и женщин на рынке труда, нивелируя дискриминацию. Одним из основных выводов статьи является то, что трудовое законодательство на практике работает неэффективно.
К допущениям стоит отнести:
\begin{itemize}
    \item Ограниченный охват исследований: ограниченный охват выборки, что может не отражать полную картину гендерных стереотипов на рынке труда в России. Часто исследователи ограничивается определенным регионом, и это может приводить к упущению некоторых аспектов. 
\end{itemize}
\url{https://cyberleninka.ru/article/n/gendernye-stereotipy-na-rynke-truda/viewer}
\section{Проблемы карьерного роста в России и зарубежных странах Кошарная Г.Б. Тарханова Е.С.}
Многие исследования могут не уделять достаточно внимания анализу и пониманию причин проблем карьерного роста женщин. Вместо этого, могут предоставляться лишь поверхностные данные о неравенстве возможностей, без раскрытия происходящих под капотом механизмов, однако в этом исследовании четко раскрыты особенности российского рынка труда. Хоть общество и относительно толерантно, однако есть немало гендерных стереотипов, поэтому кто-то может негативно воспринимать успех женщин в карьере. Также многие исследования излишне сконцентрированы на стереотипах, в отличие от этого, так как в нем рассматриваются и другие факторы, которые могут стать камнем преткновения на карьерном пути у женщин.
К допущениям стоит отнести:
\begin{itemize}
    \item Отсутствие длительного временного анализа: В исследовании отсутствует длительный временной анализ, охватывающий разные периоды и изменения в течение времени. Изменения в общественных и экономических условиях могут существенно влиять на проблемы карьерного роста женщин, и их недостаточный учет может привести к искажению результатов исследования.
\end{itemize}
\url{https://cyberleninka.ru/article/n/problemy-kariernogo-rosta-zhenschin-v-rossii-i-zarubezhnyh-stranah/viewer}
\section{Дискриминационные различия в заработной плате в России Константинова Д.С. Кудаева М.М.}
Эта статья рассказывает о различных видах дискриминации на рынке труда в России и их правовом регулировании. Также присутствует сравнения гендерной дискриминации в России с другими странами. Отмечено, что за последние 30 лет изменений в разнице доходов практически не наблюдается. Явно подчеркнута возрастная дискриминация. К допущениям стоит отнести:
\begin{itemize}
    \item Факторы объясняющие различия в заработной плате: Исследования могут оперировать с различными факторами, которые могут объяснять дискриминационные различия в заработной плате. Это могут быть факторы, связанные с образованием, опытом работы, отраслевой принадлежностью, регионом проживания и т. д. Однако выбор и оценка этих факторов может влиять на результаты исследования.
\end{itemize}
\url{https://cyberleninka.ru/article/n/diskriminatsionnye-razlichiya-v-zarabotnoy-plate-v-rossii/viewer}
\section{Ликвидация гендерного разрыва в оплате труда в STEM-отраслях как ключевая задача преодоления гендерного неравенства в странах с цифровой экономикой Задворнова Ю.С.}
В статье рассматривается проблема гендерного неравенства в STEN-отраслях, то есть среди ученых, инженеров, программистов и тд, что является колоссальной проблемой, так как женщины, которые может и задумываются работать в этой сфере боятся столкнуться с невостребованностью6 которая среди женщин гораздо выше, чем у мужчин, согласно данным, приведенным в этой статье. В результате все приходит к парадоксу, что женщины будут все реже выбирать подобные специальности, что будет еще больше усиливать неравенство. Автор приходит к логичному выводу, что человеческий капитал ограничен, поэтому необходимо его использовать по максимуму.
К допущениям стоит отнести:

\begin{itemize}
    \item Доступ к данным: Одним из основных допущений является наличие доступа к достоверным данным о заработной плате в STEM-отраслях. Эти данные могут быть собраны из различных источников, включая опросы, административные базы данных или официальную статистику. Качество данных и их полнота могут повлиять на правильность и точность результатов исследования.
\end{itemize}
\url{https://cyberleninka.ru/article/n/likvidatsiya-gendernogo-razryva-v-oplate-truda-v-stem-otraslyah-kak-klyuchevaya-zadacha-preodoleniya-gendernogo-neravenstva-v-stranah-s}
\section{Цифровой гендерный разрыв как фактор риска социальной безопасности российского общества Кисляков П.А. Шмелева Е.А.}
В статье упоминается важная проблема, а именно социальная безопасность, в особенности женщин. Рассматриваются проблемы женщин в становлении IT-специалистом, такие как стереотипы, что это не женская профессия. Также рассматривается Российский и международный опыт борьбы со стереотипами, а также вовлечению в профессию с детства. Делается вывод о том, что перед социальными институтами (семья, образование, наука, культура, общественные организации), IT-сообществом стоит задача формирования имиджа женщины, профессионально ориентированной в цифровой экономике и IT-отрасли. 
К допущениям стоит отнести:

\begin{itemize}
    \item Определение цифрового гендерного разрыва: Одним из основных допущений является определение цифрового гендерного разрыва и его влияния на социальную безопасность. Цифровой гендерный разрыв может включать в себя различия между мужчинами и женщинами в доступе к информационным и коммуникационным технологиям, уровне цифровой грамотности, использовании электронных сервисов и прочих аспектах цифровой сферы. Определение и выбор конкретных показателей цифрового гендерного разрыва будут влиять на результаты исследования. 
\end{itemize}
\url{https://cyberleninka.ru/article/n/tsifrovoy-gendernyy-razryv-kak-faktor-riska-sotsialnoy-bezopasnosti-rossiyskogo-obschestva}
\section{Гендерный подход к анализу структуры занятости по видам экономической деятельности региона Сарычева Т.В.}
В этой статье автор использует гендерный подход к анализу структуры занятости. Исследуется одна из республик, входящих в состав РФ. На основании данных исследователь приходит к выводу о том, что занятость женщин находитс примерно на уровне мужчин, средняя зарплата у женщин ощутимо ниже. Автор предлагает повысить престижность профессий и должностей, которые занимают женщины, а также внедрение гендерной статистики при найме на работу, сглаживание разницы зарплат так, чтобы равный труд приносил равный доход. К допущения можно как и ранее отнести ограниченный охват выборки, что может не отражать полную картину гендерных стереотипов на рынке труда в России, так как рассматривается конкретный регион, а РФ является многонациональной и многоконфессиональной страной. 
\url{https://cyberleninka.ru/article/n/gendernyy-podhod-k-analizu-struktury-zanyatosti-po-vidam-ekonomicheskoy-deyatelnosti-regiona/viewer}
\section{Гендерное неравенство в современной экономике России: количественный анализ проблемы Липатова Л.Н.}
В этой статье автор приходит к интересному заключению о том, что женщины представлены в органах государственной власти и гражданской службы РФ слабее мужчин, из-за чего женщинам сложнее отстаивать свои права, так как у них нет возможностей принимать решения в масштабах страны.  Также эта работа в очередной раз подтверждает факт, что мужчины в среднем получают больше женщин даже при условии, что выполняют одну и ту же работу. К допущения можно отнести:   

\begin{itemize}
    \item Многофакторный подход: Допущением является использование многофакторного подхода при анализе гендерного неравенства. Это означает учет социально-экономических, культурных и политических факторов, которые могут влиять на возникновение и поддержание гендерного неравенства в экономике. Анализ должен учитывать не только экономические показатели, но и широкий контекст социальных и политических процессов.
\end{itemize}
\url{https://cyberleninka.ru/article/n/gendernoe-neravenstvo-v-ekonomike-sovremennoy-rossii-kolichestvennyy-analiz-problemy/viewer}
\section{Почему важно развивать институты гендерного равенства в России Калабихина И.Е.}
В статье проблема гендерного равенства представлена как составляющая социально-экономического развития. В большинстве стран это одна из важнейших проблем, поэтому создается огромное количество институтов гендерного равенства. Поднимается прблема домашнего насилия в РФ, целых 40 процентов женщин с ним сталкивались. Автор приходит к выводу о там, что необходимо строить институты гендерного равенства в России на всех уровнях, а также сотрудничать с международными организации. Без решения проблемы гендерного неравенства в целом, нельзя решить частные проблемы, такие как дискриминация на рынке труда.
\url{http://genderbudgets.ru/biblio/2011_No_5_8_Kalabihina.pdf}
\section{Индекс человеческого развития и гендерное равенство: взаимообусловленность показателей Богомолова И.С. Гриненко С.В. Задорожная Е.К.}
В статье исследуется степень влияния гендерного неравенства на воспроизводство и отдачу от накопленного человеческого капитала для качественной оценки фактического уровня ИРЧП, что позволит выработать адекватную современнным социально-экономическим и общественным императивом систему управленческих взаимодействий в контексте достижения гендерного равенства как фактора экономического роста. В статье отмечается наличие в России скрытой гендерной дискриминации. При большем среднем уровне образованности и идентичнем уровнем занятости женщины получают меньше мужчин. В статье детально разобраны различные возрастные группы с помощью различных индеексов. Оказывается Россия одна из самых высокообразованных стран.
\url{https://cyberleninka.ru/article/n/indeks-chelovecheskogo-razvitiya-i-gendernoe-ravenstvo-vzaimoobuslovlennost-pokazateley/viewer}
\section{Сравнения и анализ}
Все статьи так или иначе признают гендерное неравенство, так как при равной занятости мужчины получают больше. Однако многие из статей предлаагают уникальные причины и следствия гендерного неравенства на рынке труда в России. Чаще всего упоминается вертикальная и горизонтальная сегрегация, стереотипы, низкий уровень толерантности, неэффективное трудовое законодательство, культурные и религиозные особенности и прочие причины. Лично я из всех причн выделил следующие: желание каждой отдельной семьи максимизировать полезность от труда, а также эффективное гендерное разделение труда, однако предыдущие причины тоже являются основопологающими в гендерном неравенстве на рынке труда. Некоторые статьи противоречат друг другу в вопросе образованности женского населения, так как одни исследователи основываются на том факте, что женщины более образованы (Л.Н. Липатова), а другие говорят об обратном (Ощепков А.Ю.). Однако уровень гендерного человеческого капитала действительно важный фактор для темы гендерного неравенства, так как если женщина обладают сравнимым с мужчинами гендерным капиталом, но при этом получают меньше это свидетельствует о гендерной дискриминации, а если это капитал меньше, тогда меньше доходы женщин обоснованны. Хочется отдельно ометить проблему домашнего насилия, так как она может иллюстрировать пренебрежительное отношение общества к женщинам, ведь проблема отнюдь не решена. Причинивший насилие зачастую не несет серьезной ответтственности, а как может существовать гендерное равенство на рынке труда, когда мужчина может ударить женщину, с которой находится в отношениях. На мой вгляд, нельзя решить проблему дискриминации на рынке труда, не решив проблему домашнего насилия, так как суть этих проблем в том числе заключается в недостатке уважения мужчин к женщинам, так как они считают, что мужчина не равен женщине, а может управлять и распоряжаться ею, так как он главнее. Популяризация партнерских отношений очень важна.
К общим допущениям стоит отнести:
\begin{itemize}
    \item Равномерность данных: Допущение о равномерности данных заключается в предположении о том, что дискриминация является систематической и равномерной в отношении всех групп людей.
\end{itemize}
\section{Вывод}
Женщины составляют половину мирового человеческого капитала, поэтому расширение их прав и возможностей, использование талантов и лидерства — основные элементы успеха и процветания женщин во все более конкурентном мире. Развивая умения и навыки, наращивая производительность человеческого капитала, общество может справиться с глобальными вызовами, которые несет развитие современных технологий. 
\end{document} % конец документа